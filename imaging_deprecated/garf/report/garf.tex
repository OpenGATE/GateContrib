% ================================================================
\documentclass[11pt,final,a4paper]{article}
% ==========================================
% \newif\ifpdf
% \ifx\pdfoutput\undefined \pdffalse % we are not running PDFLaTeX
% \else
% \pdfoutput=1 \pdftrue % we are running PDFLaTeX
% \fi
% ==========================================
\usepackage[a4paper,twoside,textwidth=15cm]{geometry}
\usepackage[utf8]{inputenc}
\usepackage{varioref}
\usepackage{url}
\usepackage{multicol}
%\usepackage[boxed]{algorithm}
%\usepackage{algorithmic}
\usepackage{colortbl}
\usepackage[active]{srcltx}
\usepackage{color}
%\usepackage{backref}
%\usepackage[francais]{babel}
\usepackage{picins}
\usepackage{fancyhdr}
\usepackage{textcomp}
%\usepackage{eurosans}
\usepackage[english,french]{babel}
\usepackage{eurosym}
\usepackage{picins}                       % fig ds paragraphe
\usepackage{lastpage}                     % fig ds paragraphe
\usepackage{siunitx}
%\usepackage[tight]{shorttoc}
%\usepackage{minitoc} % not by section in report

% ==========================================
%%% Math
\usepackage{amsmath,amsfonts,amssymb}
\usepackage{bm}
\usepackage{theorem}
\usepackage{stmaryrd}

% ==========================================
%%% Graphics
\usepackage{graphics}
\ifpdf
\usepackage[pdftex]{graphicx}
\else
\usepackage[ps2pdf]{graphicx}
\fi
\graphicspath{{figures/}}

\usepackage{pdfpages}

% ==========================================
%%% Page style
\setcounter{secnumdepth}{3}      % depth of numbering of sectionning commands
% \renewcommand{\chaptermark}[1]{%           % chapitre en minuscule
%   \markboth{\thechapter.\ #1}{}}
% % \renewcommand{\sectionmark}[1]{%           % section en minuscule
% %   \markboth{\thesection.\ #1}{}}
%\lhead{\slshape\footnotesize{\leftmark}}
\lhead{}
\cfoot{\thepage/\pageref{LastPage}}
\rhead{\tiny Dosimetry}
\renewcommand{\headrulewidth}{0pt}
\pagestyle{fancy}

% ==========================================
% pour avoir des zolis points au lieu de bête tiret
\newenvironment{myitemize}{%
  \begin{list}{\textbullet}{}%
    \setlength{\itemsep}{0mm}%
    \setlength{\parsep}{0mm}}%
  {\end{list}}

% ==========================================
% multiple bib
%%% Bib
%\usepackage{multibib}
%\newcites{group}{Publications}

%\usepackage[numbers,square]{natbib}
\usepackage{natbib}
%\usepackage[super,square]{natbib}
%\usepackage{bibunits}


% \makeatletter
% \adddialect\l@eng\l@french
% \makeatother

%\let\citenumfont=\tiny
%\let\citenumfont=\small

%\bibliographystyle{my_bibstyle_fr_with_url}
%\bibliographystyle{my_bibstyle_fr}

%\newcommand{\mcitep}[1]{{\scriptsize \citep{#1}}}
%\newcommand{\mcite}[1]{{\scriptsize \cite{#1}}}
%\newcommand{\mcitep}[1]{{\citep{#1}}}
%\newcommand{\mcite}[1]{{\cite{#1}}}

% splitbib bib dos not work with natbib ?
% \usepackage[export]{splitbib}
% \begin{category}[]{Toto}
%   \SBentries{monge1999,aapm2005,keall2004}
% \end{category}

% ==========================================
%% PDF configuration
% http://www.p-joeckel.de/pdflatex/)
% http://ringlord.com/publications/latex-pdf-howto/

%%%% Options of pdflatex
\ifpdf
\DeclareGraphicsExtensions{.pdf,.jpg,.mps,.png,.eps}
   % pagebackref
\usepackage[pdftex,bookmarks]{hyperref}
\pdfadjustspacing=1

%%%% Options for latex + dvips -Ppdf + ps2pdf
\else
\DeclareGraphicsExtensions{.eps,.ps}
\usepackage[ps2pdf,bookmarks,pagebackref]{hyperref}
\fi

%%%% Commun options
\hypersetup{
  linktocpage,%
  %%------------- Color Links ------------------------------
  colorlinks=true,%
  linkcolor=myred,%
  citecolor=mydarkblue,%
  urlcolor=myblue,%
  menucolor=red,%
  %%------------- Doc Info ---------------------------------
  pdftitle={SPECT simulation},%
  pdfauthor={David Sarrut},%
  %%------------ Doc View ----------------------------------}
  pdfhighlight=/P,%
  bookmarksopen=false,%
  plainpages=false,
  pdfpagelabels,
  pdfpagemode=None}

\hyperbaseurl{http://www.creatis.insa-lyon.fr/\string~dsarrut/bib/}

\definecolor{myblue}{rgb}{0,0,0.7}
\definecolor{myred}{rgb}{0.7,0,0}
\definecolor{mygreen}{rgb}{0,0.7,0}
\definecolor{mydarkblue}{rgb}{0,0,0.5}
\definecolor{mydarkgrey}{rgb}{0.3,0.3,0.3}

\def\todo{\scriptsize\fbox{\bf TODO !!}\normalsize}
\newcommand\ds[1]{\color{red}#1\color{black}}

%\newcommand{\num}[1]{n$^\circ$#1}

% ==========================================
%% section en couleur
\makeatletter
\renewcommand\section{\@startsection {section}{1}{\z@}%
        {-3.5ex \@plus -1ex \@minus -.2ex}%
        {2.3ex \@plus.2ex}%
        {\reset@font\Large\bfseries\color{mydarkblue}}}
\makeatother

%\def\u{{\bf u}}
\def\d{{\bf \phi}}
\def\x{{\bf x}}
% \def\p{{\bf p}}
% \def\r{{\bf r}}
% \def\g{{\bf g}}
% \def\A{{\bf A}}
% \def\E{{\bf E}}
% \def\h{{\bf h}}
% \def\b{{\bf b}}
% \def\k{{\bf k}}
% \def\K{{\bf K}}
% \def\P{{\bf P}}

% \def\p{{\bf p}}
% \def\q{{\bf q}}
% \def\r{{\bf r}}
% \def\c{{\bf c}}

\def\A{{\bf A}}

\newcommand{\mycite}[1]{{\color{mygreen} [#1]}}
\newcommand{\macfile}[1]{{\color{mydarkgrey} \texttt{\lstinline|#1|}}}
\newcommand{\macalias}[1]{{\texttt{\lstinline|#1|}}}

\usepackage{placeins}
\usepackage{pdfpages}
\usepackage{listings}

%%% Local Variables:
%%% mode: latex
%%% TeX-master: "rapport2006"
%%% End:


% ================================================================
\begin{document}
\thispagestyle{fancy}

\title{GARF: Gate Angular Response Function}
\author{David Sarrut}
\date{\today}

\maketitle
% ================================================================

\section{Learning ARF neural network}

Four datasets have been simulated for the four radionuclides. Each
simulation tracked $\num{e10}$ primary photons, sampled from the
energy distribution of the radionuclide. The simulation use one single
SPECT head and record photon incoming at the collimator entrance. It
lead to a dataset of about 2.3 GB of data. Russian Roulette factor is
set to 100. It tooks around 6 days of computation (on a single core).

% mac/main_dataset.mac

Neural networks for all four radionuclides were trained on those
dataset with ``config-nn-rr100-v4.json'' parameter file: 3 layers, 400
neurons by layer, 500 epoch max, 5000 batch size, 200 batch by
epoch. It tooks about 2 hours.

\begin{table}[htbp]
  \centering
  \begin{tabular}[htbp]{lclc}
    Rad & Collimator & E win & gam/decay \\\hline
    Tc99m & lehr & 2 (sc, 140) & 0.88 \\
    In111 & megp & 5 (sc 171 sc sc 245) &  1.8 \\
    Lu177 & megp & 6 (sc 113 sc sc 208 sc) & 0.17 \\
    I131 & hegp & 7 (sc 364 sc sc sc 637 722) & 1.002 \\
  \end{tabular}
  \caption{SPECT config}
  \label{tab:spect}  
\end{table}

\section{Evaluation with one example}

The reference simulation was as follows. CT thorax image resampled to
$4 \times 4 \times 4$ mm$^3$, with patient background removed (set to
Air). The density tolerance for Gate was set to $1 g/cm^3$ (very few
material). Physics list was \verb+emstandard_opt1+, cut was set to high
values in order to not track electron, except in the SPECT
crystal. The source was a voxelized source composed of 3 spheres of
about 78, 24, 20 mm diameters, with activity of 1, 1.5, 3
(normalized). About $\num{e10}$ decays was simulated for all
radionuclides (and $\num{e11}$ for Lu177), see table.

Reference analog simulation is long, more than 10 days for about
$\num{e10}$ decays. Because ARF simulation is a Variance Reduction
Technique, less primary particles is needed to reach similar
uncertainty. However, there is no simple rule to set this number. A
rule of thumb is 10 times less (see efficiency in the next
section). To compare to the analog reference simulation, image from
ARF simulation must be scaled to the targeted number of primary.

%Table simulation reference primary events + computation time

% https://www.tablesgenerator.com/ ?
% g spreadsheet with =importData("https://www.dropbox.com/s/5x5eagi2nkik77l/test.csv")
% plotly ?

\begin{table}[htbp]
  \centering
  \begin{tabular}[htbp]{lrc}
    Rad & Nb events & Time (days) \\\hline
    Tc99m & 8,584,543,002 & 35.8  \\
    In111 & 17,918,903,766 & 68.5  \\
    Lu177 & 16,872,685,913 & 64.7  \\
    I131 & 9,623,522,427 & 35.7  \\    
  \end{tabular}
  \caption{Reference simulation}
  \label{tab:ref_simu}  
\end{table}


\begin{table}[htbp]
  \centering
  \begin{tabular}[htbp]{lrc}
    Rad & Nb events & Time (d) \\\hline
    Tc99m & 884,971,913 &  4.05 days\\
    In111 & 1,847,285,705 & 8.41 days \\
    Lu177 & 1,721,704,624 & 7.6 days \\
    I131 & 1,002,419,017 & 4.3 days
  \end{tabular}
  \caption{Time for ARF simulation (see efficiency speedup in next tables).}
  \label{tab:ref_simu}  
\end{table}


\section{Results}


\begin{figure}[htbp]
  \begin{center}
    \includegraphics[width=0.9\linewidth]{Tc99m.pdf}
    \includegraphics[width=0.9\linewidth]{In111.pdf}
    \includegraphics[width=0.9\linewidth]{Lu177.pdf}
    \includegraphics[width=0.9\linewidth]{I131.pdf}
    \caption{Profiles comparison, for all windows.}
    \label{fig:profil}
  \end{center}
\end{figure}

Speed up have been computed as a mean on each slice, for all voxels
with more than 10\% max counts. Computation time is extracted from the
stats.txt file and uncertainties were computed by voxel.

\begin{table}[htbp]
  \centering

\begin{verbatim}
| Tc99m             |  eff | Analog % | ARF % |
|-------------------+------+----------+-------|
| channel 1 scatter | 21.9 | 21.9 %   | 8.6 % |
| channel 2 140 keV | 16.6 | 5.0 %    | 2.6 % |

| In111             |  eff | Analog % | ARF % |
|-------------------+------+----------+-------|
| channel 1 scatter | 39.1 | 24.7 %   | 5.0 % |
| channel 2 174 keV | 21.6 | 05.4 %   | 2.0 % |
| channel 3 scatter | 47.3 | 34.6 %   | 5.8 % |
| channel 4 scatter | 56.8 | 37.2 %   | 5.2 % |
| channel 5 245 keV | 26.5 | 06.3 %   | 1.9 % |

| Lu177             |  eff | Analog % | ARF % |
|-------------------+------+----------+-------|
| channel 1 scatter | 29.9 | 19.5 %   | 5.8 % |
| channel 2 113 keV | 24.7 | 06.3 %   | 2.3 % |
| channel 3 scatter | 34.1 | 22.49%   | 5.9 % |
| channel 4 scatter | 36.3 | 22.0 %   | 5.4 % |
| channel 5 208 keV | 27.8 | 05.8 %   | 1.8 % |
| channel 6 scatter | 87.6 | 27.4 %   | 2.8 % |

| I131              |   eff | Analog % | ARF % |
|-------------------+-------+----------+-------|
| channel 1 scatter |  90.5 | 24.1 %   | 2.2 % |
| channel 2 364 keV |  51.5 | 09.6 %   | 1.5 % |
| channel 3 scatter | 172.9 | 28.8 %   | 1.4 % |
| channel 4 scatter | 167.3 | 30.9 %   | 1.5 % |
| channel 5 scatter | 313.8 | 57.6 %   | 1.5 % |
| channel 6 637 keV | 206.7 | 38.4 %   | 1.5 % |
| channel 7 722 keV | 184.5 | 47.3 %   | 2.1 % |

\end{verbatim}
  
  \caption{Efficiency speedup ratio per channel. For ex, Lu177 in channel 3 (scatter 3) is 34 times faster with garf.}
  \label{tab:eff}  
\end{table}

\section{Conclusion}

For high activity energy windows (primary peaks), speed up is between
16 to 50, and can reach up around 200 for low activity peaks. Speed up
is better for low activity energy windows (scatter).

% ================================================================
\bibliography{all}
\bibliographystyle{unsrtnat}

\end{document}
